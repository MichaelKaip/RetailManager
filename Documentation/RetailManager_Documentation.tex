\documentclass[11pt,a4paper]{article}
\usepackage{tocloft}
\usepackage[table,xcdraw]{xcolor}
\usepackage[utf8]{inputenc}
\usepackage[english]{babel} 
\usepackage[T1]{fontenc}
\usepackage{amsmath}
\usepackage{amsfonts}
\usepackage{graphicx}
\usepackage{enumitem}
\usepackage{esvect}
\usepackage{amssymb}
\usepackage{nicefrac}
\usepackage{cancel}
\usepackage{polynom}
\usepackage{stmaryrd}
\usepackage[left=2cm,right=2cm,top=2cm,bottom=2cm]{geometry}
\author{Michael Kaip}
\title{Retail Manager Development Manual}
\usepackage{fancyhdr} % Package for Headers and footers

\pagestyle{fancy}

% Extension for amsmath matrix environment - matrix | vector
\makeatletter
	\renewcommand*\env@matrix[1][*\c@MaxMatrixCols c]{%
  	\hskip -\arraycolsep
  	\let\@ifnextchar\new@ifnextchar
  	\array{#1}}
\makeatother


%Extension for roman numbers
\newcommand{\uproman}[1]{\uppercase\expandafter{\romannumeral#1}}
\newcommand{\lowroman}[1]{\romannumeral#1\relax}

%%%%%%%%%%%%%%%%%%%%%%%%%%%%%%%%%%%%%%%%%%%%%%%%%%%%%%%%%%%

\begin{document}
\maketitle
\pagenumbering{gobble}
\newpage
\tableofcontents
\newpage
\pagenumbering{arabic}

\section{Introduction}

\subsection{Project Summary}
The goal of tkhis project is to build a dektop app that runs a cash register, handles inventory and manages an entire retail store. Creating and implementing a \textbf{WebAPI layer}, will allow the whole project to grow. This lauyer will be able to serve each kind of application (desktop, mobile, web, ...).
\newpage

\section{Initial Plan}

\subsection{Outline}
The App is going to be build as a MVP (Minimum Viable Product) that can be expanded to cover all of the features, which are needed over time - so it can grow into a full featured application. First step is getting all of the major pieces set up, including:



\subsection{Technologies}

\begin{minipage}[t]{0.5\linewidth}\vspace{0pt}
\begin{itemize}
\item Unit Testing
\item Dependency Injection
\item WPF
\item MVVM with Caliburn Micro
\item ASP.NET MVC (Web Frontend)
\item .NET Framework
\item .NET Core 3.0
\item SSDT - SQL Server Data Tools
\item Git
\item Azure DevOps
\end{itemize}
\end{minipage}
\begin{minipage}[t]{0.5\linewidth}\vspace{0pt}
\begin{itemize}
\item Async
\item Reporting
\item WebAPI
\item Logging
\item Data Validation
\item HTML
\item CSS
\item JavaScript
\item Authentication
\end{itemize}
\end{minipage}
\newpage

\section{Initial Setup in Visual Studio}

\begin{enumerate}
\item Setting up a Git-Repository, including README, GitIgnore (for VS) and License
\item Creating a \textbf{Blank Solution}: Other Project Types $\to$ Blank Solution\\
Such type of sulution isn't language specific.
\end{enumerate}
\newpage

\section{Creating a WebAPI with Authentification}

\begin{enumerate}
\item Adding new Project to the Solution:\\\\
Web $\to$ ASP.NET Web Application (.NET Framework) $\to$ WebAPI\\\\
Add folders and references for:
\begin{itemize}
\item MVC
\item Web API
\end{itemize}
Change Authentication to
\begin{itemize}
\item Individual User Accounts
\end{itemize}
\item Upgrading all NuGet-Packages
\end{enumerate}
\subsection{Indentity Configuration}
App\_Start $\to$ IdentityConfig.cs\\\\
In there are some settings for setting up the WebAPI, especially for authentication:
\begin{itemize}
\item UserValidator
\item PasswordValidator
\end{itemize}
\subsection{Getting authorized for development}

\subsubsection{Postman}
The following calls has to be applied in the given order:
\begin{enumerate}
\item \textbf{POST}\\\\
\boxed{\includegraphics[scale=0.5]{pic01}}\\\\
Creates a new user account and stores this information into the user database.\\
If \textbf{Status: 200 OK}, username and password has been succesfully created.
\item \textbf{GET}\\\\
\boxed{\includegraphics[scale=0.5]{pic02}}\\\\
It will return an \textbf{access\_token} which is, by default, valid for 14 days. Token is needed for all further interaction with the server. Can be also configured for shorter valid periods.
\item \textbf{POST}\\\\
\boxed{\includegraphics[scale=0.5]{pic03}}
\end{enumerate}

\subsubsection{Getting User Information}
In order to get the Identity of users returned, some changes have to be implemented. Through this it's becomes possible to apply different accesibility rules, based on the user-group a certain user is part of.
\begin{enumerate}
\item \boxed{RMDataManager.Controllers.ValuesController}\\\\
\boxed{\includegraphics[scale=0.46]{pic04}}
\end{enumerate}
\newpage

\section{Installing and configuring SWAGGER}
SWAGGER is an API documentation and demonstration tool.

\subsection{Installing SWAGGER}

\begin{enumerate}
\item NuGet-Manager\\\\
\boxed{\includegraphics[scale=0.38]{pic05}}\\\\
Adds a SWAGGER to WebAPI-Projects.
\item Starting SWAGGER\\\\
\boxed{\includegraphics[scale=0.38]{pic06}}\\\\
\end{enumerate}

\subsection{Channging the configuration of SWAGGER}
\boxed{RMDataManager.App\_Start.SwaggerConfig.cs}
\begin{enumerate}
\item Changing title\\\\
\boxed{\includegraphics[scale=0.38]{pic07}}
\item Enabling propper printing of ducuments\\\\
\boxed{\includegraphics[scale=0.38]{pic08}}
\item Treating Enums as Strings\\\\
\boxed{\includegraphics[scale=0.348]{pic09}}
\item Changing document title\\\\
\boxed{\includegraphics[scale=0.348]{pic10}}
\end{enumerate}

\subsection{Adding OAuth ability}
\begin{enumerate}
\item Enabling token endpoint allowance in the SWAGGER documentation
\begin{enumerate}
\item Adding a new Class to \boxed{RMDataManager.App\_Start}\\\\
\boxed{\includegraphics[scale=0.328]{pic11}}\newpage
\item Implementing the required Interface\\\\
\boxed{\includegraphics[scale=0.328]{pic12}}\\
\item Applying it to SwaggerConfig.cs\\\\
\boxed{\includegraphics[scale=0.328]{pic13}}\\
\item Logging into the application using SWAGGER and get the token\\\\
\boxed{\includegraphics[scale=0.366]{pic14}}\\
\end{enumerate}
\newpage
\item Enabling to paste in the bearer token in order to authorize restricted commands
\begin{enumerate}
\item Adding a new Class to \boxed{RMDataManager.App\_Start}\\\\
\boxed{\includegraphics[scale=0.376]{pic15}}\\
\item Implementing the required Interface\\\\
\boxed{\includegraphics[scale=0.376]{pic16}}\\
\item Applying it to SwaggerConfig.cs\\\\
\boxed{\includegraphics[scale=0.376]{pic17}}\\
\item Get user information from the application via SWAGGER using the token\\\\
\boxed{\includegraphics[scale=0.376]{pic18}}\\
\end{enumerate}

\end{enumerate}
\newpage

\section{SQL Database Setup}

\subsection{Adding new Database Project to the solution}
\boxed{\includegraphics[scale=0.405]{pic19}}\\

\subsection{Adding several folders to the project}
\boxed{\includegraphics[scale=1.0]{pic20}} \newpage

\subsection{Creating a profile and publishing the Database}
\begin{enumerate}
\item \boxed{RightClick~on~RMDatabase \rightarrow Publish \rightarrow Edit \rightarrow Browse}\\\\
\boxed{\includegraphics[scale=0.4]{pic21}}\\
\item Naming and saving profile to PublishLocations\\\\
\boxed{\includegraphics[scale=0.4]{pic22}}\\
\end{enumerate}

\section{WPF with MVVM Project Setup}

\subsection{Adding the WPF Project to the solution}
\boxed{\includegraphics[scale=0.375]{pic23}}\\

\subsection{Changing the Assambly Name to the name of the solution in Properties}
\boxed{\includegraphics[scale=0.375]{pic24}}\\\\
Also set project as the default startup-project.
\newpage

\subsection{Adding Caliburn Micro MVVM-Framework}
Add NuGet-Package to references.

\subsubsection{Adding the folder structure for the MVVM-Framework}
\begin{center}\boxed{\includegraphics[scale=0.5]{pic25}}\end{center}

\subsubsection{Adding a new ShellViewModel class and a SchellView window}
\begin{center}\boxed{\includegraphics[scale=0.5]{pic26}}\end{center}

\subsubsection{Adding a Bootstrapper class to DesktopUI}
\begin{minipage}[t]{0.32\linewidth}\vspace{0pt}
\boxed{\includegraphics[scale=0.5]{pic27}}
\end{minipage}
\begin{minipage}[t]{0.68\linewidth}\vspace{0pt}
\boxed{\includegraphics[scale=0.35]{pic28}}
\end{minipage}

\subsubsection{Removing StartUpURI from App.xaml and adding a new Ressource Dictionary}
\begin{center}\boxed{\includegraphics[scale=0.48]{pic29}}\end{center}
\textbf{MainWindow.xaml can be deleted afterwards!!!}
\newpage

\section{Dependency Injection in WPF}

\subsection{SimpleContainer in Caliburn Micro}
Caliburn.Micro comes pre-bundled with a Dependency Injection container called SimpleContainer. A dependency injection container is an object that is used to hold dependency mappings for use later in an app via Dependency Injection. Dependency Injection is actually a pattern typically using the container element instead of manual service mapping.

\subsubsection{Implementing SimpleContainer in Bootstrapper.cs}
\begin{center}\boxed{\includegraphics[scale=0.48]{pic30}}\end{center}

\subsection{Overriding Configure() Method for the container}
\begin{center}\boxed{\includegraphics[scale=0.48]{pic31}}\end{center}
\newpage

\section{Datamodel - planning and setup}

\subsection{Planning the Register}
\begin{center}\boxed{\includegraphics[scale=0.46]{pic32}}\end{center}
\newpage

\subsection{SQL Database Table Creation}

\begin{minipage}[t]{0.5\linewidth}\vspace{0pt}
\boxed{\textbf{User.sql}}\\\\
\includegraphics[scale=0.35]{pic33}
\end{minipage}
\begin{minipage}[t]{0.5\linewidth}\vspace{0pt}
\boxed{\textbf{Product.sql}}\\\\
\includegraphics[scale=0.35]{pic34}
\end{minipage}
\\\\
\begin{minipage}[t]{0.6\linewidth}\vspace{0pt}
\boxed{\textbf{Sale.sql}}\\\\
\includegraphics[scale=0.35]{pic35}
\end{minipage}
\begin{minipage}[t]{0.4\linewidth}\vspace{0pt}
\boxed{\textbf{SaleDetail.sql}}\\\\
\includegraphics[scale=0.35]{pic36}
\end{minipage}
\\\\
\begin{minipage}[t]{0.5\linewidth}\vspace{0pt}
\boxed{\textbf{Inventory.sql}}\\\\
\includegraphics[scale=0.35]{pic37}
\end{minipage}
\begin{minipage}[t]{0.5\linewidth}\vspace{0pt}
!!! \boxed{\textbf{Publishing Tables}} !!! \\\\
\includegraphics[scale=0.35]{pic38}
\end{minipage}

\section{WPF Login Form Creation}

\subsection{Inheritance from the conductor class in Caliburn Micro}
\begin{center}\boxed{\includegraphics[scale=0.55]{pic39}}\end{center}
Conductor is a base class which inherits from Screen. Its responsibility is to conduct other objects by managing an active item and maintain a strict lifecycle of this conducted item. The conductor exists in multiple variants such as the one item conductor simple called Conductor, the multiple item conductors such as \boxed{Conductor.Collection.OneActive and Conductor.Collection.AllActive}.

\subsection{Implementing the menue bar}
\begin{minipage}[t]{0.5\linewidth}\vspace{0pt}
\boxed{\includegraphics[scale=0.34]{pic40}}
\end{minipage}
\begin{minipage}[t]{0.5\linewidth}\vspace{0pt}
\boxed{\includegraphics[scale=0.34]{pic41}}
\end{minipage}

\subsection{Adding a UserControl}

\subsubsection{Adding a class LoginViewModel (public) and UserControl LoginView}
\begin{center}\boxed{\includegraphics[scale=0.55]{pic42}}\end{center}

\subsubsection{Designing the UserControl}
\begin{minipage}[t]{0.5\linewidth}\vspace{0pt}
\boxed{\includegraphics[scale=0.34]{pic44}}
\end{minipage}
\begin{minipage}[t]{0.5\linewidth}\vspace{0pt}
\boxed{\includegraphics[scale=0.34]{pic45}}
\end{minipage}

\subsubsection{Activating the LoginView on startup in the ShellView}
\begin{minipage}[t]{0.5\linewidth}\vspace{0pt}
\boxed{\includegraphics[scale=0.29]{pic46}}
\end{minipage}
\begin{minipage}[t]{0.5\linewidth}\vspace{0pt}
\boxed{\includegraphics[scale=0.29]{pic47}}
\end{minipage}\\\\\\
Sealed is used to restrict the users from inheriting. A class can be sealed by using the sealed keyword or a single method. The keyword tells the compiler that class or method cannot be extended. No class can be derived from a sealed class.
\newpage

\subsubsection{Implementing LoginViewModel.cs}
\boxed{\includegraphics[scale=0.9]{pic51}}

\subsubsection{Connecting the LoginViewModel to Caliburn.Micro}
\begin{enumerate}
\item Adding a helper class (PasswordBoxHelper.cs) to RMDesktopUI
\begin{center}\boxed{\includegraphics[scale=0.55]{pic49}}\end{center}
\item Adding some lines to the constructor of Bootstrapper.cs
\begin{center}\boxed{\includegraphics[scale=0.495]{pic50}}\end{center}
\end{enumerate}
See the example on stackoverflow.com...
\newpage

\section{Wiring up the WPF Login Form}
Connecting the login form button to the authentication API endpoint (/token). Gets back the bearer token or an exception if failed.

\subsection{Implementing a class AuthenticatedUser.cs}
Holds the information for already authenticated users.\\\\
\boxed{\includegraphics[scale=0.68]{pic53}}

\subsection{Implementing a  helper class to handle API call interactions}
\boxed{\includegraphics[scale=0.68]{pic52}}
\newpage

\subsection{Implementing a Interface IAPIHelper.cs}
Needed for dependency injection, in order to add it to the Configure() method in Bootstrapper.cs\\\\
\boxed{\includegraphics[scale=0.68]{pic54}}

\subsection{Adding <appsettings> to App.Config}
\boxed{\includegraphics[scale=0.638]{pic55}}

\subsection{Adding APIHelper and IAPIHelper to the container in Bootstrapper.cs}
\boxed{\includegraphics[scale=0.638]{pic56}}

\subsection{Applying some changings to LoginViewModel.cs}

\subsubsection{Adding a new private property as a backing field}
\boxed{\includegraphics[scale=0.638]{pic57}}

\subsubsection{Adding a constructor and initializing the property from within}
\boxed{\includegraphics[scale=0.638]{pic58}}

\subsubsection{Implementig the Login() method}
\boxed{\includegraphics[scale=0.524]{pic59}}

\subsection{Enabling the solution to start multiple projects}
\boxed{\includegraphics[scale=0.638]{pic60}}\\\\
\boxed{\includegraphics[scale=0.524]{pic61}}\\\\
\boxed{\includegraphics[scale=0.524]{pic62}}

\newpage
\section{Login Form Error Handling}

\subsection{Displaying an login  error message within the login form}
\boxed{\includegraphics[scale=0.524]{pic63}}\\\\
Through \boxed{Visibility} the funktionality of collapsing the error message space in case of no error is going to be displayed is added. In case of an error the field expands and the error is going to be displayed. To make it work, first the \boxed{BooleanToVisibilityConverter} has to be added to the RessourceDictionary in App.xaml:\\\\
\boxed{\includegraphics[scale=0.524]{pic64}}\\\\
Afterwards properties for \boxed{ErrorMessage} and \boxed{IsErrorVisible} has to be implemented in LoginViewModel.cs:\\\\
\boxed{\includegraphics[scale=0.624]{pic65}}\\\\
And finally the Login() method has to be changed in the way, that in case of an exception, the exception message is stored into the \boxed{ErrorMessage} property:\\\\
\boxed{\includegraphics[scale=0.625]{pic66}}\\\\

\section{Getting User Data}
Getting all information about a logged in user from the database. This information is going to be stored in an singleton object and therefore can be used as long as the user is logged in. Once he's logged out, the corresponding object which holds the user information is going to be destroyed.\\\\
In order to make it work, following things has to be done:

\begin{itemize}
\item implementing a stored procedure
\item implementing a model to store the data in the API
\item creating an API endpoint
\item implementing a method that calls the endpoint from the WPF application
\item implementing a model that hold data in the WPF application
\end{itemize}

\subsection{Adding a class library for the API}
The aim of this library is to provide data access. It is seperate from the WPF application an so it knows nothing about the databse, nor have access to it. Putting code not directly into the API enables re-usability.\\\\
\boxed{\includegraphics[scale=0.625]{pic67}}

\subsubsection{Installing Dapper}
Dapper is an object-relational mapping (ORM) product for the Microsoft .NET platform: it provides a framework for mapping an object-oriented domain model to a traditional relational database. Its purpose is to relieve the developer from a significant portion of relational data persistence-related programming tasks.\\\\
Add NuGet-Package to references...

\subsubsection{The SqlDataAccess class}
\boxed{\includegraphics[scale=0.65]{pic68}}
\newpage

\subsubsection{The UserData class}
This class contains methods and the logic on \textbf{how} to get data.

\begin{enumerate}
\item Adding a stored Prodecure to \boxed{RMData.dbo.StoredProcedures}\\\\
\boxed{\includegraphics[scale=0.62]{pic69}}\\\\

\item Adding a UserModel class\\\\
\boxed{\includegraphics[scale=0.62]{pic70}}\\\\
\newpage

\item Adding the UserData class\\\\
\boxed{\includegraphics[scale=0.57]{pic71}}
\end{enumerate}

\subsubsection{Adding a UserController.cs to RMDataManager.Controllers}
\boxed{\includegraphics[scale=0.69]{pic72}}\\\\
Implementing the UserController:\\\\
\boxed{\includegraphics[scale=0.49]{pic74}}\\\\
Publishing the stored procedure to the database:\\\\
\boxed{\includegraphics[scale=0.9]{pic75}}\\\\
Adding a user for testing:\\\\
\begin{minipage}[t]{0.5\linewidth}\vspace{0pt}
\boxed{\includegraphics[scale=0.43]{pic76}}\\\\
\end{minipage}
\begin{minipage}[t]{0.5\linewidth}\vspace{0pt}
\boxed{\includegraphics[scale=0.43]{pic77}}
\end{minipage}
Getting the connection string from the database and adding it to Web.config:\\\\
\boxed{\includegraphics[scale=0.43]{pic78}}\\\\
\boxed{\includegraphics[scale=0.43]{pic79}}\\\\
Now that there's an entry in the user database, the application can be started and use SWAGGER to run and test the GET command in \boxed{Values/GET}\\\\
\textbf{Next step is to implement the functionality to get back the information about the user automatically, once he logged in successfully.}

\subsection{Implementing data capturing from the UI side}

\subsubsection{Adding a new class library}
The aim of this library is for user inface support.\\\\
\boxed{\includegraphics[scale=0.43]{pic80}}

\subsubsection{The LoggedInUserModel class}
\boxed{\includegraphics[scale=0.5]{pic81}}

\subsubsection{Moving helpers and Models}
\boxed{\includegraphics[scale=0.43]{pic82}}\\\\
Namespaces has to be ajusted and missing references added afterwards.

\subsubsection{Extracting an Interface and adding to simpleContainer}
\begin{minipage}[t]{0.5\linewidth}\vspace{0pt}
\boxed{\includegraphics[scale=0.3]{pic83}}\\\\
\end{minipage}
\begin{minipage}[t]{0.5\linewidth}\vspace{0pt}
\boxed{\includegraphics[scale=0.3]{pic84}}\\\\
\end{minipage}

\subsubsection{Implementing a Method GetLoggedInUserInfo in APIHelper.cs}
\boxed{\includegraphics[scale=0.468]{pic85}}\\\\
In order to make it work, some more changes has to be done in the same class:\\\\
\boxed{\includegraphics[scale=0.465]{pic86}}\\\\
Finally the LogIn() method in the LoginViewModel has to be extended:\\\\
\boxed{\includegraphics[scale=0.494]{pic87}}\\\\

\section{Sales Page Creation}
\begin{enumerate}
\item Adding a SalesViewModel.cs and SalesView.xaml (UserControl) to RMDesktopUI
\item Designing the sales page
\end{enumerate}

\subsection{Designing the Sales Page}
\begin{minipage}[t]{0.5\linewidth}\vspace{0pt}
\boxed{\includegraphics[scale=0.42]{pic90}}
\end{minipage}
\begin{minipage}[t]{0.5\linewidth}\vspace{0pt}
\boxed{\includegraphics[scale=0.42]{pic88}}
\boxed{\includegraphics[scale=0.42]{pic89}}\\\\
\end{minipage}

\subsection{Implementing SalesViewModel.cs}
\begin{minipage}[t]{0.5\linewidth}\vspace{0pt}
\boxed{\includegraphics[scale=0.42]{pic91}}\\
\boxed{\includegraphics[scale=0.42]{pic92}}\\\\
\end{minipage}
\begin{minipage}[t]{0.5\linewidth}\vspace{0pt}
\boxed{\includegraphics[scale=0.42]{pic93}}
\end{minipage}
Just scaffolds out the methods needed in this class. Will be implemented later...
\newpage

\section{Event Aggregation in WPF}
\subsection{Wiring up an event aggregator to enable inter-form communication}
\begin{minipage}[t]{0.5\linewidth}\vspace{0pt}
\boxed{\includegraphics[scale=0.42]{pic94}}\\\\
\end{minipage}
\begin{minipage}[t]{0.5\linewidth}\vspace{0pt}
\boxed{\includegraphics[scale=0.35]{pic95}}
\end{minipage}
That's an example of the usefullness of DI. It doesn't have to be instatiated or set up. Because it's assigned in simpleContainer, everthing is going to be done behind the scenes in the framework and then it can be used everywhere in that class.\\\\
\begin{minipage}[t]{0.5\linewidth}\vspace{0pt}
\boxed{\includegraphics[scale=0.33]{pic96}}\\\\
\end{minipage}
\begin{minipage}[t]{0.5\linewidth}\vspace{0pt}
\boxed{\includegraphics[scale=0.3]{pic98}}\\\\
\boxed{\includegraphics[scale=0.35]{pic97}}
\end{minipage}
Next step, is to listen for the event on the other side (ShellViewModel.cs).\\\\
\boxed{\includegraphics[scale=0.35]{pic99}}

\section{Displaying Product data}
Gaining the available items for sale from the database, through the API into the WPF-Project.

\subsection{Adding a new column to the product table}
\boxed{\includegraphics[scale=0.62]{pic100}}


\subsection{Getting data into the API}

\subsubsection{Adding a new stored procedure for data request}
\boxed{\includegraphics[scale=0.6]{pic101}}

\subsubsection{Creating an endpoint in the API for the products}
\begin{enumerate}
\item First a new to controller has to be added to \boxed{RMDataManager.Controllers} \\\\
\boxed{\includegraphics[scale=0.63]{pic102}}
\newpage
\item Implementing the controller\\\\
\boxed{\includegraphics[scale=0.63]{pic104}}
\item Adding a ProductModel to \boxed{RMDataManager.Library}\\\\
\boxed{\includegraphics[scale=0.63]{pic103}}
\newpage
\item Adding a new class ProductData.cs to \boxed{RMDataManager.Library.DataAccess}\\\\
\boxed{\includegraphics[scale=0.45]{pic105}}
\end{enumerate}

\subsubsection{ Connecting from the WPF side}
\begin{enumerate}
\item Applying some little changes in \boxed{APIHelper.cs}\\\\
\boxed{\includegraphics[scale=0.45]{pic106}}
\item Adding a new product endpoint and extracting an Interface from it\\\\
\boxed{\includegraphics[scale=0.4]{pic107}}
\newpage
\item Adding a new property to \boxed{IAPIHelper.cs} and \boxed{APIHelper.cs}\\\\
\boxed{\includegraphics[scale=0.4]{pic108}}\\\\
\boxed{\includegraphics[scale=0.4]{pic109}}
\item Adding the product model to the UI\\\\
\boxed{\includegraphics[scale=0.4]{pic110}}
\item Adding it to Bootstrapper.cs\\\\
\boxed{\includegraphics[scale=0.4]{pic111}}
\item Adding a constructor to SalesViewModel.cs and injecting it from there as well as implementing a method for loading product data\\\\
\boxed{\includegraphics[scale=0.4]{pic112}}\\\\
\boxed{\includegraphics[scale=0.4]{pic113}}
\item Modifying how the SalesView is gong to display this products\\\\
\boxed{\includegraphics[scale=0.4]{pic114}}
\end{enumerate}

\section{Wiring up WPF Shopping Cart}
\begin{enumerate}
\item Adding a new Property to \boxed{SalesViewModel.cs}\\\\
\boxed{\includegraphics[scale=0.41]{pic115}}
\item Now the property can be used in \boxed{SalesView.xaml}\\\\
\boxed{\includegraphics[scale=0.41]{pic116}}
\item Making sure the selected quantity is valid\\\\
\boxed{\includegraphics[scale=0.41]{pic117}}\\\\
\boxed{\includegraphics[scale=0.41]{pic118}}
\item Adding the selected product to the cart
\begin{enumerate}
\item Adding a new Model to \boxed{RMDektopUI.Library.Models}\\\\
Calculates the the quantities of items in the cart.\\\\
\boxed{\includegraphics[scale=0.41]{pic119}}
\item Changint the type of the Cart property to the type of this new model\\\\
\boxed{\includegraphics[scale=0.41]{pic120}}\\\\
This is goint to cange the UI...\\
\item Adding a \boxed{ListBox.ItemTemplate} to the Cart\\\\
\boxed{\includegraphics[scale=0.41]{pic122}}\newpage
\item ICreating an instance of \boxed{CartItemModel} and add the model to the Cart\\\\
\boxed{\includegraphics[scale=0.41]{pic121}}
\item Initializing a new Bindinglist for the Cart\\\\
\boxed{\includegraphics[scale=0.41]{pic123}}
\item Showing the quantity of the selected items in the cart\\\\
\boxed{\includegraphics[scale=0.41]{pic124}}\\\\
\boxed{\includegraphics[scale=0.41]{pic125}}
\end{enumerate}
\item Calculating \boxed{SubTotal}\\\\
\boxed{\includegraphics[scale=0.41]{pic126}}\\\\
\boxed{\includegraphics[scale=0.41]{pic127}}\newpage
\item Modifying \boxed{QuantityInStock} based upon how much items has been already added to the cart\\\\
\boxed{\includegraphics[scale=0.41]{pic128}}
\item Updating the selected quantity of an existing item in the cart\\\\
\boxed{\includegraphics[scale=0.41]{pic129}}
\end{enumerate}

\section{Modifying SQL, the API and WPF to add Taxes}


\end{document}