\documentclass[11pt,a4paper]{article}
\usepackage{tocloft}
\usepackage[table,xcdraw]{xcolor}
\usepackage[utf8]{inputenc}
\usepackage[english]{babel} 
\usepackage[T1]{fontenc}
\usepackage{amsmath}
\usepackage{amsfonts}
\usepackage{graphicx}
\usepackage{esvect}
\usepackage{amssymb}
\usepackage{nicefrac}
\usepackage{cancel}
\usepackage{polynom}
\usepackage{stmaryrd}
\usepackage[left=2cm,right=2cm,top=2cm,bottom=2cm]{geometry}
\author{Michael Kaip}
\title{Retail Manager Development Manual}
\usepackage{fancyhdr} % Package for Headers and footers

\pagestyle{fancy}

% Extension for amsmath matrix environment - matrix | vector
\makeatletter
	\renewcommand*\env@matrix[1][*\c@MaxMatrixCols c]{%
  	\hskip -\arraycolsep
  	\let\@ifnextchar\new@ifnextchar
  	\array{#1}}
\makeatother


%Extension for roman numbers
\newcommand{\uproman}[1]{\uppercase\expandafter{\romannumeral#1}}
\newcommand{\lowroman}[1]{\romannumeral#1\relax}

%%%%%%%%%%%%%%%%%%%%%%%%%%%%%%%%%%%%%%%%%%%%%%%%%%%%%%%%%%%

\begin{document}
\maketitle
\pagenumbering{gobble}
\newpage
\tableofcontents
\newpage
\pagenumbering{arabic}

\section{Introduction}

\subsection{Project Summary}
The goal of tkhis project is to build a dektop app that runs a cash register, handles inventory and manages an entire retail store. Creating and implementing a \textbf{WebAPI layer}, will allow the whole project to grow. This lauyer will be able to serve each kind of application (desktop, mobile, web, ...).
\newpage

\section{Initial Plan}

\subsection{Outline}
The App is going to be build as a MVP (Minimum Viable Product) that can be expanded to cover all of the features, which are needed over time - so it can grow into a full featured application. First step is getting all of the major pieces set up, including:

\begin{itemize}
\item Git on Azure DevOps
\item SQL Database (SSDT)
\item WebAPI with Authentification
\item WPF application that can log into the API
\end{itemize}

\subsection{Technologies}

\begin{minipage}[t]{0.5\linewidth}\vspace{0pt}
\begin{itemize}
\item Unit Testing
\item Dependency Injection
\item WPF
\item MVVM with Caliburn Micro
\item ASP.NET MVC (Web Frontend)
\item .NET Framework
\item .NET Core 3.0
\item SSDT - SQL Server Data Tools
\item Git
\item Azure DevOps
\end{itemize}
\end{minipage}
\begin{minipage}[t]{0.5\linewidth}\vspace{0pt}
\begin{itemize}
\item Async
\item Reporting
\item WebAPI
\item Logging
\item Data Validation
\item HTML
\item CSS
\item JavaScript
\item Authentication
\end{itemize}
\end{minipage}
\newpage

\section{Initial Setup in Visual Studio}

\begin{enumerate}
\item Setting up a Git-Repository, including README, GitIgnore (for VS) and License
\item Creating a \textbf{Blank Solution}: Other Project Types $\to$ Blank Solution\\
Such type of sulution isn't language specific.
\end{enumerate}
\newpage

\section{Creating a WebAPI with Authentification}

\begin{enumerate}
\item Adding new Project to the Solution:\\\\
Web $\to$ ASP.NET Web Application (.NET Framework) $\to$ WebAPI\\\\
Add folders and references for:
\begin{itemize}
\item MVC
\item Web API
\end{itemize}
Change Authentication to
\begin{itemize}
\item Individual User Accounts
\end{itemize}
\item Upgrading all NuGet-Packages
\end{enumerate}
\subsection{Indentity Configuration}
App\_Start $\to$ IdentityConfig.cs\\\\
In there are some settings for setting up the WebAPI, especially for authentication:
\begin{itemize}
\item UserValidator
\item PasswordValidator
\end{itemize}
\subsection{Getting authorized for development}

\subsubsection{Postman}
The following calls has to be applied in the given order:
\begin{enumerate}
\item \textbf{POST}\\\\
\boxed{\includegraphics[scale=0.5]{pic01}}\\\\
Creates a new user account and stores this information into the user database.\\
If \textbf{Status: 200 OK}, username and password has been succesfully created.
\item \textbf{GET}\\\\
\boxed{\includegraphics[scale=0.5]{pic02}}\\\\
It will return an \textbf{access\_token} which is, by default, valid for 14 days. Token is needed for all further interaction with the server. Can be also configured for shorter valid periods.
\item \textbf{POST}\\\\
\boxed{\includegraphics[scale=0.5]{pic03}}
\end{enumerate}

\subsubsection{Getting User Information}
In order to get the Identity of users returned, some changes have to be implemented. Through this it's becomes possible to apply different accesibility rules, based on the user-group a certain user is part of.
\begin{enumerate}
\item \boxed{RMDataManager.Controllers.ValuesController}\\\\
\boxed{\includegraphics[scale=0.46]{pic04}}
\end{enumerate}
\newpage

\section{Installing and configuring SWAGGER}
SWAGGER is an API documentation and demonstration tool.

\subsection{Installing SWAGGER}

\begin{enumerate}
\item NuGet-Manager\\\\
\boxed{\includegraphics[scale=0.38]{pic05}}\\\\
Adds a SWAGGER to WebAPI-Projects.
\item Starting SWAGGER\\\\
\boxed{\includegraphics[scale=0.38]{pic06}}\\\\
\end{enumerate}

\subsection{Channging the configuration of SWAGGER}
\boxed{RMDataManager.App\_Start.SwaggerConfig.cs}
\begin{enumerate}
\item Changing title\\\\
\boxed{\includegraphics[scale=0.38]{pic07}}
\item Enabling propper printing of ducuments\\\\
\boxed{\includegraphics[scale=0.38]{pic08}}
\item Treating Enums as Strings\\\\
\boxed{\includegraphics[scale=0.348]{pic09}}
\item Changing document title\\\\
\boxed{\includegraphics[scale=0.348]{pic10}}
\end{enumerate}

\subsection{Adding OAuth ability}
\begin{enumerate}
\item Enabling token endpoint allowance in the SWAGGER documentation
\begin{enumerate}
\item Adding a new Class to \boxed{RMDataManager.App\_Start}\\\\
\boxed{\includegraphics[scale=0.328]{pic11}}\newpage
\item Implementing the required Interface\\\\
\boxed{\includegraphics[scale=0.328]{pic12}}\\
\item Applying it to SwaggerConfig.cs\\\\
\boxed{\includegraphics[scale=0.328]{pic13}}\\
\item Logging into the application using SWAGGER and get the token\\\\
\boxed{\includegraphics[scale=0.366]{pic14}}\\
\end{enumerate}
\newpage
\item Enabling to paste in the bearer token in order to authorize restricted commands
\begin{enumerate}
\item Adding a new Class to \boxed{RMDataManager.App\_Start}\\\\
\boxed{\includegraphics[scale=0.376]{pic15}}\\
\item Implementing the required Interface\\\\
\boxed{\includegraphics[scale=0.376]{pic16}}\\
\item Applying it to SwaggerConfig.cs\\\\
\boxed{\includegraphics[scale=0.376]{pic17}}\\
\item Get user information from the application via SWAGGER using the token\\\\
\boxed{\includegraphics[scale=0.376]{pic18}}\\
\end{enumerate}

\end{enumerate}
\newpage

\section{SQL Database Setup}

\subsection{Adding new Database Project to the solution}
\boxed{\includegraphics[scale=0.405]{pic19}}\\

\subsection{Adding several folders to the project}
\boxed{\includegraphics[scale=1.0]{pic20}} \newpage

\subsection{Creating a profile and publishing the Database}
\begin{enumerate}
\item \boxed{RightClick~on~RMDatabase \rightarrow Publish \rightarrow Edit \rightarrow Browse}\\\\
\boxed{\includegraphics[scale=0.4]{pic21}}\\
\item Naming and saving profile to PublishLocations\\\\
\boxed{\includegraphics[scale=0.4]{pic22}}\\
\end{enumerate}

\section{WPF with MVVM Project Setup}

\subsection{Adding the WPF Project to the solution}
\boxed{\includegraphics[scale=0.375]{pic23}}\\

\subsection{Changing the Assambly Name to the name of the solution in Properties}
\boxed{\includegraphics[scale=0.375]{pic24}}\\\\
Also set project as the default startup-project.
\newpage

\subsection{Adding Caliburn Micro MVVM-Framework}
Add NuGet-Package to references.

\subsubsection{Adding the folder structure for the MVVM-Framework}
\begin{center}\boxed{\includegraphics[scale=0.5]{pic25}}\end{center}

\subsubsection{Adding a new ShellViewModel class and a SchellView window}
\begin{center}\boxed{\includegraphics[scale=0.5]{pic26}}\end{center}

\subsubsection{Adding a Bootstrapper class to DesktopUI}
\begin{minipage}[t]{0.32\linewidth}\vspace{0pt}
\boxed{\includegraphics[scale=0.5]{pic27}}
\end{minipage}
\begin{minipage}[t]{0.68\linewidth}\vspace{0pt}
\boxed{\includegraphics[scale=0.35]{pic28}}
\end{minipage}

\subsubsection{Removing StartUpURI from App.xaml and adding a new Ressource Dictionary}
\begin{center}\boxed{\includegraphics[scale=0.48]{pic29}}\end{center}
\textbf{MainWindow.xaml can be deleted afterwards!!!}
\newpage

\section{Dependency Injection in WPF}

\subsection{SimpleContainer in Caliburn Micro}
Caliburn.Micro comes pre-bundled with a Dependency Injection container called SimpleContainer. A dependency injection container is an object that is used to hold dependency mappings for use later in an app via Dependency Injection. Dependency Injection is actually a pattern typically using the container element instead of manual service mapping.

\subsubsection{Implementing SimpleContainer in Bootstrapper.cs}
\begin{center}\boxed{\includegraphics[scale=0.48]{pic30}}\end{center}

\subsection{Overriding Configure() Method for the container}
\begin{center}\boxed{\includegraphics[scale=0.48]{pic31}}\end{center}
\newpage

\section{Datamodel - planning and setup}

\subsection{Planning the Register}
\begin{center}\boxed{\includegraphics[scale=0.46]{pic32}}\end{center}
\newpage

\subsection{SQL Database Table Creation}

\begin{minipage}[t]{0.5\linewidth}\vspace{0pt}
\boxed{\textbf{User.sql}}\\\\
\includegraphics[scale=0.35]{pic33}
\end{minipage}
\begin{minipage}[t]{0.5\linewidth}\vspace{0pt}
\boxed{\textbf{Product.sql}}\\\\
\includegraphics[scale=0.35]{pic34}
\end{minipage}
\\\\
\begin{minipage}[t]{0.6\linewidth}\vspace{0pt}
\boxed{\textbf{Sale.sql}}\\\\
\includegraphics[scale=0.35]{pic35}
\end{minipage}
\begin{minipage}[t]{0.4\linewidth}\vspace{0pt}
\boxed{\textbf{SaleDetail.sql}}\\\\
\includegraphics[scale=0.35]{pic36}
\end{minipage}
\\\\
\begin{minipage}[t]{0.5\linewidth}\vspace{0pt}
\boxed{\textbf{Inventory.sql}}\\\\
\includegraphics[scale=0.35]{pic37}
\end{minipage}
\begin{minipage}[t]{0.5\linewidth}\vspace{0pt}
!!! \boxed{\textbf{Publishing Tables}} !!! \\\\
\includegraphics[scale=0.35]{pic38}
\end{minipage}

\section{WPF Login Form Creation}

\subsection{Inheritance from the conductor class in Caliburn Micro}
\begin{center}\boxed{\includegraphics[scale=0.55]{pic39}}\end{center}
Conductor is a base class which inherits from Screen. Its responsibility is to conduct other objects by managing an active item and maintain a strict lifecycle of this conducted item. The conductor exists in multiple variants such as the one item conductor simple called Conductor, the multiple item conductors such as \boxed{Conductor.Collection.OneActive and Conductor.Collection.AllActive}.

\subsection{Implementing the menue bar}
\begin{minipage}[t]{0.5\linewidth}\vspace{0pt}
\boxed{\includegraphics[scale=0.34]{pic40}}
\end{minipage}
\begin{minipage}[t]{0.5\linewidth}\vspace{0pt}
\boxed{\includegraphics[scale=0.34]{pic41}}
\end{minipage}

\subsection{Adding a UserControl}

\subsubsection{Adding a class LoginViewModel (public) and UserControl LoginView}
\begin{center}\boxed{\includegraphics[scale=0.55]{pic42}}\end{center}

\subsubsection{Designing the UserControl}
\begin{minipage}[t]{0.5\linewidth}\vspace{0pt}
\boxed{\includegraphics[scale=0.34]{pic44}}
\end{minipage}
\begin{minipage}[t]{0.5\linewidth}\vspace{0pt}
\boxed{\includegraphics[scale=0.34]{pic45}}
\end{minipage}

\subsubsection{Activating the LoginView on startup in the ShellView}
\begin{minipage}[t]{0.5\linewidth}\vspace{0pt}
\boxed{\includegraphics[scale=0.29]{pic46}}
\end{minipage}
\begin{minipage}[t]{0.5\linewidth}\vspace{0pt}
\boxed{\includegraphics[scale=0.29]{pic47}}
\end{minipage}\\\\\\
Sealed is used to restrict the users from inheriting. A class can be sealed by using the sealed keyword or a single method. The keyword tells the compiler that class or method cannot be extended. No class can be derived from a sealed class.
\newpage

\subsubsection{Implementing LoginViewModel.cs}
\boxed{\includegraphics[scale=0.9]{pic51}}

\subsubsection{Connecting the LoginViewModel to Caliburn.Micro}
\begin{enumerate}
\item Adding a helper class (PasswordBoxHelper.cs) to RMDesktopUI
\begin{center}\boxed{\includegraphics[scale=0.55]{pic49}}\end{center}
\item Adding some lines to the constructor of Bootstrapper.cs
\begin{center}\boxed{\includegraphics[scale=0.495]{pic50}}\end{center}
\end{enumerate}
See the example on stackoverflow.com...
\section{Wiring up the WPF Login Form to the API}

\section{Login Form Error Handling}

\section{Getting User Data}

\section{Sales Page Creation}

\section{Event Aggregation in WPF}

\section{Displaying Product data}

\section{Wiring up WPF Shopping Cart}

\section{Modifying SQL, the API and WPF to add Taxes}


\end{document}